%%%%%%%%%%%%%%%%%%%%%%%%%%%%%%%%%%%%%%%%%%%%%%%%%%%%%%%%%%%%%%%%%%%%%%%%%%%
% Acteurs
%%%%%%%%%%%%%%%%%%%%%%%%%%%%%%%%%%%%%%%%%%%%%%%%%%%%%%%%%%%%%%%%%%%%%%%%%%%
\section{Rôles}
\includeDiagram{Usecases/Hierarchie}{Hiérarchie des rôles}

\role{Utilisateur client graphique}{uiUser}{C'est une personne ayant accès aux fonctionnalités graphiques du logiciel SGBag. Les autres rôles \og Utilisateurs \fg en dérivent.}

\role{Utilisateur configuration}
{confUser}
{C'est une personne qui manipule l'interface de configuration. Seul le responsable technique possède les autorisations pour modifier ou créer des configurations.}

\role{Utilisateur simulation}
{simUser}
{C'est une personne qui manipule l'interface de simulation. Le responsable technique et le superviseur en font partie.}

\role{Utilisateur exploitation}
{expUser}
{C'est une personne qui manipule l'interface de maintenance. Le superviseur et l'informaticien sont les deux employés pouvant accéder à cette interface.}

\role{Utilisateur maintenance}
{mntUser}
{C'est une personne qui manipule l'interface de maintenance. Le superviseur et l'informaticien sont les deux employés pouvant accéder à cette interface.}

\role{Utilisateur gestion réclamation}
{recUser}
{C'est une personne qui gère les problèmes survenus en cas de perte ou de dégradation des bagages. Elle peut accéder au dossier des litiges ainsi qu'au trajet des bagages.}

\role{Objet dynamique}
{obj}
{Objet commandable en mouvement, à distance, tel qu'un tapis, un chariot, \ldots}

\role{Capteur actif}
{sensor}
{Capteur interrompant le système lorsqu'il doit transmettre une information (alarme, \ldots).}

\role{Capteur passif}
{passiveSensors}
{Capteur dont la valeur doit être lue à intervalle régulier (caméra, \ldots).}

\role{Horloge}
{clock}
{Une horloge, déclenchant des événement à une fréquence donnée.}

\role{Serveur base de données}
{bdd}
{La base de donnée du système de gestion de bagages contient toutes les données de configuration et de gestion de l'application.}

\role{PDA Technicien}
{pda}
{C'est le PDA\footnote{\textsl{Personal Digital Assistant}} de la personne qui intervient en cas de problème.}


%%%%%%%%%%%%%%%%%%%%%%%%%%%%%%%%%%%%%%%%%%%%%%%%%%%%%%%%%%%%%%%%%%%%%%%%%%%
% Cas d'utilisation
%%%%%%%%%%%%%%%%%%%%%%%%%%%%%%%%%%%%%%%%%%%%%%%%%%%%%%%%%%%%%%%%%%%%%%%%%%%

\section{Cas d'utilisation}

%%%%%%%%%%%%%%%%%%%%%%%%%%%%%%%%%%%%%%%%%%%%%%%%%%%%%%%%%%%%%%%%%
\subsection{Besoins communs}
\includeDiagram{Usecases/BesoinsCommuns}{Cas d'utilisations : Besoins communs}

\subsection{Visualiser un objet statique}
\user{L'utilisateur du client graphique}{uiUser} ou \user{L'utilisateur de l'interface de configuration}{confUser} peut sélectionner, zoomer ou dézoomer sur un objet de haut niveau.

\subsection{Effectuer des actions sur l'historique}
\user{L'utilisateur du client graphique}{uiUser} peut annuler les actions qu'il vient de réaliser ou à l'inverse les reproduire après les avoir annulé.

%%%%%%%%%%%%%%%%%%%%%%%%%%%%%%%%%%%%%%%%%%%%%%%%%%%%%%%%%%%%%%%%%
\subsection{Configuration}
\includeDiagram{Usecases/Configuration}{Cas d'utilisations : Configuration}

\subsubsection{Effectuer des opérations sur un élément}
\user{L'utilisateur de la configuration}{confUser} peut visualiser, ajouter, déplacer, lier aux autres éléments, paramétrer ou supprimer un élément de la configuration courante.

\subsubsection{Simuler la configuration en cours}
\user{L'utilisateur de la configuration}{confUser} peut ouvrir l'interface de simulation à partir de l'interface de configuration. Les paramètres de la simulation sont ceux de la configuration en cours.

\subsubsection{Gérer la persistance d'une configuration}
\user{L'utilisateur de la configuration}{confUser} peut enregistrer, enregistrer sous, charger, créer, supprimer ou dupliquer la configuration dans la \user{base de donnée}{bdd}.

%%%%%%%%%%%%%%%%%%%%%%%%%%%%%%%%%%%%%%%%%%%%%%%%%%%%%%%%%%%%%%%%%
\subsection{Simulation}
\includeDiagram{Usecases/Simulation}{Cas d'utilisations : Simulation}

\subsubsection{Charger une configuration} 
\user{L'utilisateur de la simulation}{simUser} peut définir la configuration d'aéroport utilisée pour la simulation, à partir de la \user{base de donnée}{bdd}.

\subsubsection{Gérer la liste des vols}
\user{L'utilisateur de la simulation}{simUser} peut ajouter, retirer, paramétrer des vols à simuler.

\subsubsection{Changer le mode de simulation}
\user{L'utilisateur de la simulation}{simUser} peut basculer entre mode manuel et automatique.

\subsubsection{Agir manuellement sur les éléments}
\user{L'utilisateur de la simulation}{simUser} peut sélectionner, paramétrer, mettre en marche/arrêt un objet dynamique.

\subsubsection{Effectuer des opérations sur l'avancement de la simulation}
\user{L'utilisateur de la simulation}{simUser} peut démarrer, stopper, mettre en pause, modifier la vitesse de la simulation et par conséquent modifier les paramètres de fonctionnement de \user{l'horloge}{clock}.

\subsubsection{Gérer les événements}
\user{L'utilisateur de la simulation}{simUser}Créer, modifier, supprimer, visualiser, activer/désactiver des événements.

\subsection{Gérer la persistance d'une simulation}
\user{L'utilisateur de la simulation}{simUser} peut enregistrer, enregistrer sous, charger, créer, supprimer ou dupliquer la configuration de la simulation dans la \user{base de donnée}{bdd}

\subsection{Mettre à jour l'état du système ponctuellement}
\user{L'utilisateur de la simulation}{simUser} peut mettre à jour l'état du système, \textsl{i.e.} les positions, l'état des différents objets, redessiner \textsl{etc.}

\subsection{Mettre à jour l'état du système périodiquement}
\user{L'horloge}{clock} peut mettre à jour l'état du système.

\subsection{Visualiser un objet dynamique}
\user{L'utilisateur de l'interface de simulation}{simUser} peut sélectionner, zoomer ou dézoomer sur un chariot ou un avion.
%%%%%%%%%%%%%%%%%%%%%%%%%%%%%%%%%%%%%%%%%%%%%%%%%%%%%%%%%%%%%%%%%
\subsection{Exploitation}
\includeDiagram{Usecases/Exploitation}{Cas d'utilisations : Exploitation}

\subsubsection{Charger une configuration}
\user{L'utilisateur exploitation}{simUser} peut définir la configuration d'aéroport utilisée pour l'exploitation, à partir de la \user{base de donnée}{bdd}.

\subsection{Gérer la persistance des fichiers de journalisation}
\user{L'utilisateur exploitation}{expUser} peut enregistrer, enregistrer sous, charger, crée, supprimer ou dupliquer un fichier de journalisation à partir, ou dans, la \user{base de donnée}{bdd}.

\subsection{Effectuer un arrêt d'urgence}
\user{L'utilisateur exploitation}{expUser} peut effectuer un arrêt d'urgence. Commande l'arrêt d'urgence de tous les \user{objets dynamiques}{obj}.

\subsubsection{Effectuer une opération protégée sur un éléments}
\user{L'utilisateur exploitation}{expUser} peut paramétrer, arrêter ou démarrer \user{l'objet dynamique}{obj}.

\subsubsection{Acheminer automatiquement les bagages}
À chaque ticks d'\user{horloge}{clock}, et lorsque c'est nécessaire, le système détermine le chemin que chaque \user{objet dynamique}{obj} doit emprunter.

\subsection{Mettre à jour l'état du système périodiquement}
\user{L'horloge}{clock} peut mettre à jour l'état du système en consultant les \user{capteurs passifs}{passiveSensors}.

\subsection{Mettre à jour l'état du système ponctuellement}
\user{Un capteur actif}{sensor} peut mettre à jour l'état du système.

\subsection{Visualiser un objet dynamique}
\user{L'utilisateur exploitation}{expUser} peut sélectionner, zoomer ou dézoomer sur un chariot ou un avion.

%%%%%%%%%%%%%%%%%%%%%%%%%%%%%%%%%%%%%%%%%%%%%%%%%%%%%%%%%%%%%%%%%
\subsection{Maintenance}
\includeDiagram{Usecases/Maintenance}{Cas d'utilisations : Maintenance}

\subsubsection{Effectuer des opérations non protégées sur un élément}
\user{L'utilisateur de la maintenance}{expUser} peut effectuer des opérations habituellement interdites par le système (\textsl{e.g.} ajouter dans un conteneur déjà surchargé) sur un  \user{objet dynamique}{obj}.

\subsubsection{Visualiser les résultats}
\user{L'utilisateur de la maintenance}{mntUser} peut visualiser le comportement du système (via une constatation sur le terrain) à la suite d'opération d'exploitation sur le système.

\subsubsection{Gérer la persistance des fiches électroniques d'interventions et des fichiers de journaux}
\user{L'utilisateur de la maintenance}{mntUser} peut créer, modifier, supprimer, ou renommer un fichier journal ou une fiche d'intervention de la \user{base de donnée}{bdd}.

\subsubsection{Gérer les interventions}
\user{L'utilisateur de la maintenance}{mntUser} peut, en cas de problème, assigner la résolution d'un problème à un technicien via \user{son PDA technicien}{pda} et créer un fichier électronique qui contient un rapport d'intervention en \user{base de donnée}{bdd}.

%%%%%%%%%%%%%%%%%%%%%%%%%%%%%%%%%%%%%%%%%%%%%%%%%%%%%%%%%%%%%%%%%
\subsection{Réclamation}
\includeDiagram{Usecases/Reclamations}{Cas d'utilisations : Réclamation}

\subsubsection{Gérer un dossier de litige}
En cas de litige,  \user{l'utilisateur gestion réclamations}{recUser} peut (perte ou dégradation d'un bagage), ouvrir, modifier, visualiser ou fermer un dossier de litige. Il peut aussi créer un nouveau couple identifiant/Mot de passe voyageur.

\subsubsection	{Se renseigner sur un bagage}
\user{L'utilisateur gestion réclamation}{recUser} peut visualiser le trajet d'un bagage pendant son traitement par le système de gestion à partir des données de \user{la base de donnée}{bdd}.

\subsubsection{Gérer la persistance des dossiers de litiges}
\user{L'utilisateur gestion réclamation}{mntUser} peut charger, sauvegarder ou supprime un fichier journal ou une fiche d'intervention de la \user{base de donnée}{bdd}.

