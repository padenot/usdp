%%%%%%%%%%%%%%%%%%%%%%%%%%%%%%%%%%%%%%%%%%%%%%%%%%%%%%%%%%%%%%%%%%%%%%%%%%%
% Acteurs
%%%%%%%%%%%%%%%%%%%%%%%%%%%%%%%%%%%%%%%%%%%%%%%%%%%%%%%%%%%%%%%%%%%%%%%%%%%
\section{Acteurs et rôles}
\subsection{Utilisateur client graphique}
C'est une personne ayant accès aux fonctionnalités graphiques du logiciel SGBag. Les autres rôles \og Utilisateurs \fg en dérivent.
% TODO : inclure le diagramme d'héritage des rôles utilisateurs

\subsection{Utilisateur configuration}
C'est une personne qui manipule l'interface de configuration. Seul le responsable technique possède les autorisations pour modifier ou créer des configurations.

\subsection{Utilisateur simulation}
C'est une personne qui manipule l'interface de simulation. Le responsable technique et le superviseur en font partie.

\subsection{Utilisateur exploitation}
C'est une personne qui manipule l'interface de maintenance. Le superviseur et l'informaticien sont les deux employés pouvant accéder à cette interface.

\subsection{Utilisateur maintenance}
C'est une personne qui manipule l'interface de maintenance. Le superviseur et l'informaticien sont les deux employés pouvant accéder à cette interface.

\subsection{Utilisateur gestion réclamation}
C'est une personne qui gère les problèmes survenus en cas de perte ou de dégradation des bagages. Elle peut accéder au dossier des litiges ainsi qu'au trajet des bagages.

\subsection{Objet actif}
Objet commandable à distance, tel qu'un tapis, un chariot, \ldots

\subsection{Capteur actif}
Capteur interrompant le système lorsqu'il doit transmettre une information (alarme, \ldots).

\subsection{Capteur passif}
Capteur dont la valeur doit être lue à intervalle régulier (caméra, \ldots).

\subsection{Horloge}
Une horloge, déclenchant des événement à une fréquence donnée.

\subsection{Base de données}
La base de donnée du système de gestion de bagages contient toutes les tables accessibles par l'application (\textsl{e.g} tblConfiguration, tblSimulation, \textsl{etc,} )
{\huge TODO : supprimer ?}

\subsection{BdD Exploitation}
Permet de stocker, entre autres, l'historique des déplacement, informations reçues du système, etc.

\subsection{BdD Configuration}
Permet de stocker les configurations créées.

\subsection{Technicien}
C'est la personne qui intervient en cas de problème. Elle peut consulter et modifier la \textsl{fiche électronique d'intervention}.




%%%%%%%%%%%%%%%%%%%%%%%%%%%%%%%%%%%%%%%%%%%%%%%%%%%%%%%%%%%%%%%%%%%%%%%%%%%
% Cas d'utilisation
%%%%%%%%%%%%%%%%%%%%%%%%%%%%%%%%%%%%%%%%%%%%%%%%%%%%%%%%%%%%%%%%%%%%%%%%%%%

\section{Cas d'utilisation}

%%%%%%%%%%%%%%%%%%%%%%%%%%%%%%%%%%%%%%%%%%%%%%%%%%%%%%%%%%%%%%%%%
\subsection{Besoins communs}
\subsubsection{Mettre à jour l'état du système}
Prendre en compte les changements intervenus dans l'environnement pour conserver une représentation des données à jour.
\subsubsection{Visualisation statique}
Visualiser les objets propres à la configuration, donc immobiles (hall, terminal, tapis, \ldots).
\subsubsection{Visualisation dynamique}
Visualiser les objets propres à la simulation, donc mobiles (chariots, avions).
\subsubsection{Gérer la persistance}
{\huge ATTENTION, INCOHERENT}

%%%%%%%%%%%%%%%%%%%%%%%%%%%%%%%%%%%%%%%%%%%%%%%%%%%%%%%%%%%%%%%%%
\subsection{Configuration}
\subsubsection{Ajouter un élément}
Créer un nouvel élément et l'insérer dans la configuration. 

\subsubsection{Simuler}
Ouvrir l'interface de simulation à partir de l'interface de configuration. Les paramètres de la simulation sont ceux de la configuration en cours.

\subsubsection{Manipuler un élément existant}
Déplacer, lier, supprimer ou paramètrer un élément sélectionné.

\subsubsection{Fermer la configuration}
Quitter l'interface de configuration en demandant éventuellement de sauvegarder la configuration courrante. {\huge ICI, Problème !}

%%%%%%%%%%%%%%%%%%%%%%%%%%%%%%%%%%%%%%%%%%%%%%%%%%%%%%%%%%%%%%%%%
\subsection{Simulation}
\subsubsection{Charger une configuration}
Définir la configuration d'aéroport utilisée pour la simulation, à partir de la BdD Configuration.
\subsubsection{Gérer la liste des vols}
Ajouter, retirer, paramétrer des vols à simuler.
\subsubsection{Changer le mode de simulation}
Basculer entre mode manuel et automatique.
\subsubsection{Agir sur les éléments}
Sélectionner, paramétrer, mettre en marche/arrêt.
\subsubsection{Effectuer des opérations sur l'avancement de la simulation}
Démarrer, stopper, mettre en pause, modifier la vitesse de la simulation.
\subsubsection{Gérer les événements}
Créer, modifier, supprimer, visualiser, activer/désactiver des événements.
\subsubsection{Déclencher les événements}
Déclencher les événements lorsqu'ils doivent l'être.
%%%%%%%%%%%%%%%%%%%%%%%%%%%%%%%%%%%%%%%%%%%%%%%%%%%%%%%%%%%%%%%%%
\subsection{Exploitation}
\subsubsection{Effectuer une opération protégée}
Paramétrer les éléments, les arrêter, les démarrer.
\subsubsection{Acheminer automatiquement les bagages}
{\huge TODO : à déplacer}
\subsubsection{Arrêter d'urgence le système}
Commande l'arrêt d'urgence de tous les objets contrôlables à distance.
%%%%%%%%%%%%%%%%%%%%%%%%%%%%%%%%%%%%%%%%%%%%%%%%%%%%%%%%%%%%%%%%%
\subsection{Maintenance}
\subsubsection{Visualiser les résultats}
Visualiser le comportement du système (via une constatation sur le terrain) à la suite d'opération d'epxloitation sur le système.

\subsubsection{Gérer les interventions}
En cas de problème, assigner la résolution d'un problème à un technicien et créer un fichier électronique qui contient un rapport d'intervention.

%%%%%%%%%%%%%%%%%%%%%%%%%%%%%%%%%%%%%%%%%%%%%%%%%%%%%%%%%%%%%%%%%
\subsection{Réclamation}
\subsubsection{Gérer un dossier de litige}
En cas de litige (perte ou dégradation d'un bagage), créer, ouvrir, modifier, visualiser ou fermer un dossier de litige.

\subsubsection{Se renseigner sur un bagage}
Visualiser le trajet d'un bagage pendant son traitement par le système de gestion.

{\huge TODO : Définir la notion d'élément ??}


