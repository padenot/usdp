\section{Définitions}
\nDefinition{Object métier haut niveau :}
{Terminal, hall,  zone de maintenance, zone de contrôle, voie de garage, zone d'embarquement, zone de déchargement, zone de retrait, zone d'enregistrement des bagages}


\nDefinition{Objets métier de bas niveau :}
{Guichet, chariot, carroussel, tapis roulant, capteur actif, capteur passif,  train de wagonnets, tobogan, portique, circuit de déplacement des chariots}

\nDefinition{Elément :}{Objet de haut ou de bas niveau}

\nDefinition{Mode de simulation}
{
Il peut être automatique ou manuel. Lorsqu'il est automatique, l'intervention humaine n'est nécessaire qu'en cas de problème ou de blocage du système.

En Mode manuel, l'intervention d'un opérateur est nécessaire à chaque décision que le système doit prendre (\textsl{e.g.} embranchement, départ d'un bagage )
}

\textbf{Remarque :} Un élément de haut niveau peut-être mis en marche ou arrêté. Ainsi, tous les éléments inclus seront récursivement mis en marche ou arrêté.

\section{Liste des besoins}
\subsection{Besoins communs}
\nBesoin
{Visualiser un objet dynamique}
{Chariot, avion}
{Sélectionner l'objet, zoomer/ dézoomer sur un objet métier de haut niveau}

\nBesoin
{Visualiser un objet statique}
{Objets métiers de haut niveau}
{Sélectionner l'objet, zoomer/dézoomer}

\nBesoin
{Effectuer des actions sur l'historique}
{}
{Annuler/rétablir une opération}

\subsection{Configuration}
\setcounter{cntBesoins}{1}

\nBesoin
{Effectuer des opérations sur un élément}
{Objets métiers haut et bas niveau}
{Visualiser, ajouter, déplacer, lier aux autres éléments, paramètrer, supprimer}

\nBesoin
{Simuler la configuration en cours}
{}
{}

\nBesoin
{Gérer la persistance d'une configuration}
{}
{Enregistrer, enregistrer sous, charger, créer, supprimer, dupliquer, modifier}

\subsection{Simulation}
\setcounter{cntBesoins}{1}

\nBesoin
{Charger une configuration pour la simulation}
{}
{}

\nBesoin
{Gérer la liste des vols}
{}
{Créer, supprimer, modifier un vol}

\nBesoin
{Changer le mode de simulation entre automatique et manuel}
{}
{}

\nBesoin
{Agir manuellement sur les éléments visualisés}
{Objets métiers haut et bas niveau}
{Sélectionner, parametrer, mettre en arrêt/marche}

\nBesoin
{Effectuer des opérations sur l'avancement de la simulation}
{}
{Stopper/démarrer/mettre en pause, changer la vitesse de simulation (changer le top d'horloge)}

\nBesoin
{Gérer les événements}
{Arrivée bagage, panne, départ avion, \textsl{etc.}}
{Créer, modifier, supprimer, visualiser, activer/désactiver}

\nBesoin
{Déclencher les événements}
{Arrivée bagage, panne, départ avion, \textsl{etc.}}
{}

\nBesoin
{Gérer la persistance d'une simulation}
{}
{Enregistrer, enregistrer sous, charger, créer, supprimer, dupliquer, modifier}

\nBesoin
{Mettre à jour l'état du système suite à un tick d'horloge}
{}
{}

\nBesoin
{Mettre à jour l'état du système suite à une alarme}
{}
{}

\subsection{Exploitation}
\setcounter{cntBesoins}{1}

\nBesoin
{Effectuer un arrêt d'urgence}
{}{}

\nBesoin
{Acheminer automatiquement les bagages}
{}{}

\nBesoin
{Stocker l'historique}
{Opérations effectuées, informations reçues}
{}

\nBesoin
{Effectuer des opérations protégées sur des éléments}
{Paramétrer, mettre en arrêt/marche}
{Objets métiers haut et bas niveau, bagage}

\nBesoin
{Gérer la persistance d'une configuration et des fichiers de logs}
{}
{Charger}

\nBesoin
{Mettre à jour l'état du système suite à un tick d'horloge}
{}{}

\nBesoin
{Mettre à jour l'état du système suite à une alarme}
{}{}

\subsection{Maintenance}
\setcounter{cntBesoins}{1}

\nBesoin
{Manipuler l'ensemble du système}
{}
{Les opérations d'exploitation sans restriction}

\nBesoin
{Visualiser les résultats/problèmes}
{}{}

\nBesoin
{Gérer les interventions}
{}
{Assigner la résolution d'un problème à un technicien, créer un fichier électronique qui contient un rapport d'intervention}

\nBesoin
{Gérer la persistance des fiches électroniques d'intervention et des fichiers de journaux.}
{}
{Enregistrer, charger, créer, supprimer, modifier}

\subsection{Réclamation}
\setcounter{cntBesoins}{1}

\nBesoin
{Gérer un dossier de  litige}
{}
{Créer/modifier/visualiser/fermer}

\nBesoin
{Créer un nouvel identifiant de voyageur}
{}
{Créer un numéro d'identification et un mot de passe}

\nBesoin
{Se renseigner sur un bagage}
{}
{Visualiser le trajet d'un bagage,  suivre état de traitement d'un bagage}`

\nBesoin
{Gérer la persistance des dossiers de litiges}
{}
{Enregistrer, charger, supprimer}
